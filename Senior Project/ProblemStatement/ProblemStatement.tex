\documentclass [10pt, titlepage]{article}
\usepackage [margin=0.75in]{geometry}
\begin{document}
\title{CS461 Problem Statement}
\author{Brian Huang}
\date{\today}
\maketitle

Currently to check for Ebola doctors must use contact sensors such as thermometers, which is slow and can
potentially infect them as well as other patients. With a large volume of people this method is extremely inefficient and can potentially increase the spread of the disease to 
other people. People who are infected with the Ebola virus have elevated core body temperatures, which is easy to detect using a contact sensor. Since Ebola
is easy to detect with a thermometer why risk infecting care workers by having them take pateint data. The key next step to this problem is automation. This project
looks to solve that problem.\par


The goal of this project is to build a machine in collaboration with Medecins Sans Frontieres that can detect signs of ebola with a contact sensor.
This allows for less contact with the doctors, reducing risk for any unnecessary infection. Using a machine to detect for infection will also increase
patient throughput as well as freeing up doctors to focus on other pressing matters. If this machine is successfully implemented and put into practice,
many hospital's overall efficentcy and effectivness will improve.\par


The machine will be built by two different capstone groups at Oregon State University. The physical sensor system will be desined and created by the MIME
capstone team, while the software will be programmed by the Computer Science capstone team. The goal of the software is to create a model that can predict
the patients core body temperature based on the data that is gathered by the stand sensor.\par

\end{document}
